\documentclass[11pt]{article}

% Language setting
% Replace `english' with e.g. `spanish' to change the document language
\usepackage[english]{babel}

% Set page size and margins
% Replace `letterpaper' with `a4paper' for UK/EU standard size
\usepackage[a4paper,top=2cm,bottom=2cm,left=2cm,right=2cm,marginparwidth=1.75cm]{geometry}

% Useful packages
\usepackage{amsmath,amssymb,amsfonts}
\usepackage{graphicx}
\usepackage[colorlinks=true, allcolors=blue]{hyperref}

\title{\textbf{Generation of optimized structures using Particle Swarn Optimization (PSO)}}
\author{Antoine GISSLER}
\date{January 10th, 2023}

\begin{document}
\maketitle

\section{Introduction}
ouais \cite{original}
Les simulations de systèmes moléculaires, qu'ils soient complexes ou non, nécessitent l'utilisation de configurations initiales. 
Faire une introduction sur la nécessité de devoir générer des configurations par la théorie dans le cas de conditions expérimentales exotiques (on ne peut pas se baser sur des expériences précédentes car il n'y en a pas lol)
\section{Particle Swarn Optimization}

\section{Comparison to other generation methods}

\section{Conclusion}

\bibliographystyle{acm}
\bibliography{articles}

\end{document}